\documentclass[12pt,a4paper]{article}

\input{../../preamble_files/packages}
\input{../../preamble_files/scriptr}
\input{../../preamble_files/siunits}
\input{../../preamble_files/vectors}
\input{../../preamble_files/figures}
\input{../../preamble_files/references}
\input{../../preamble_files/empheq}

\pagestyle{fancy}
\lhead{Richard Whitehill}
\chead{PHYS 631 -- HW L}
\rhead{03/10/22}
\cfoot{\thepage \hspace{1pt} of \pageref{LastPage}}

\newcommand{\prob}[2]{\textbf{#1)} #2}

\setlength{\parskip}{\baselineskip}
\setlength{\parindent}{0pt}

\begin{document}

\prob{4.4}{A point charge $q$ is situated a large distance $r$ from a neutral atom of polarizability $\alpha$. Find the force of attraction between them.}

The charge $q$ serves as the external field which polarizes the neutral atom. The dipole moment of the atom from this field is given as
\begin{align*}
p = \alpha\qty(\frac{1}{4\pi\epsilon_0}\frac{q}{r^2})
\end{align*}
Thus, we can calculate the field produced by the induced dipole
\begin{align*}
E = \frac{1}{4\pi\epsilon_0}\frac{1}{r^3}\qty(\frac{2\alpha q}{4\pi\epsilon_0 r^2})
\end{align*}
and the force on $q$ from this field 
\begin{eqbox}
F = qE_{\rm atom} = \frac{2\alpha q^2}{4\pi\epsilon_0 r^5}
\end{eqbox}

\prob{4.6}{A (perfect) dipole $\va{p}$ is situated a distance $z$ above an infinite grounded conducting plane. The dipole makes an angle $\theta$ with the perpendicular to the plane. Find the torque on $\va{p}$. If the dipole is free to rotate, in what orientation will it come to rest?}

The torque on the dipole is given as
\begin{align*}
\va{N} = \va{p} \cross \va{E}
\end{align*}
We can produce an equivalent set up by placing a mirror dipole on the opposite side of the infinite conducting plane, which gives us the electric field from the plane. That is,
\begin{align*}
\va{p} &= p\qty[\cos{\theta}\rhat + \sin{\theta}\thhat] \\
\va{E} &= \frac{p}{4\pi\epsilon_0(2z)^3}\qty(2\cos{\theta}\rhat + \sin{\theta}\thhat)
\end{align*}
Hence, carrying out the cross product and simplifying the result,
\begin{eqbox}
\va{N} = -\frac{p^2\sin(2\theta)}{4\pi\epsilon_0\qty(16z^3)}\phihat
\end{eqbox}
From this we can see that the dipole ``attempts'' to orient itself parallel to the field. If $-\pi/2 < \theta < \pi/2$ the dipole will point away from the plane. Otherwise the dipole will rotate such that it points towards the plane. 

\end{document}
