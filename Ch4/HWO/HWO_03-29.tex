\documentclass[12pt,a4paper]{article}

\input{../../preamble_files/packages}
\input{../../preamble_files/scriptr}
\input{../../preamble_files/siunits}
\input{../../preamble_files/vectors}
\input{../../preamble_files/figures}
\input{../../preamble_files/references}
\input{../../preamble_files/empheq}

\pagestyle{fancy}
\lhead{Richard Whitehill}
\chead{PHYS 631 -- HW O}
\rhead{03/29/22}
\cfoot{\thepage \hspace{1pt} of \pageref{LastPage}}

\newcommand{\prob}[2]{\textbf{#1)} #2}

\setlength{\parskip}{\baselineskip}
\setlength{\parindent}{0pt}

\begin{document}

\prob{4.23}{Find the field inside a sphere of linear dielectric material in an otherwise uniform electric field $\va{E}_0$ by the following method of successive approximations: First pretend the field inside is just $\va{E}_0$, and use Eq. 4.30 to write down the resulting polarization $\va{P}_0$. This polarization generates a field of its own, $\va{E}_1$, which in turn modifies the polarization by an amount $\va{P}_1$, which futher changes the field by an amount $\va{E}_2$, and so on. The resulting field is $\va{E}_0 + \va{E}_1 + \va{E}_2 + \hdots$. Sum the series, and compare your answer with Eq. 4.49.}

If the material is immersed in a medium with a uniform field $\va{E}_0$, then the polarization from this field is found to be $\va{P}_0 = \epsilon_0\chi_e\va{E}_0$. Then we can find the field $\va{E}_1$ from this polarization as $\va{E}_1 = -\frac{1}{3\epsilon_0}\va{P}_1 = -\frac{\chi_e}{3}\va{E}_0$. This produces another polarization $\va{P}_1 = \epsilon_0\chi_e\va{E}_1 = -\frac{\epsilon_0\chi_e^2}{3}\va{E}_0$, which again produces a change to the electric field $\va{E}_2 = -\frac{1}{3\epsilon_0}\va{P}_1 = \frac{\chi_e^2}{9}\va{E}_0$. This pattern will continue, giving the total electric field as an infinite sum over the small corrections to the electric field after each successive ``polarization'':
\begin{align*}
\va{E} = \sum_{n=0}^{\infty} \va{E}_n = \sum_{n=0}^{\infty} \qty(-\frac{\chi_e}{3})^n\va{E}_0 = \frac{1}{1+\chi_e/3}\va{E}_0
\end{align*}
Recall the relation between susceptibility and relative permeability: $\chi_e = \epsilon_r - 1$. Thus,
\begin{eqbox}
\va{E} = \frac{3}{3+\chi_e}\va{E}_0 = \frac{3}{2+\epsilon_r}\va{E}_0
\end{eqbox}

\prob{4.39}{A conducting sphere at potential $V_0$ is half embedded in linear dielectric material of susceptibility $\chi_e$, which occupies the region $z < 0$. \textit{Claim:} the potential everywhere is exactly the same as it would have been in the absence of the dielectric! Check this claim as follows:}

a) Write down the formula for the proposed potential $V(r)$, in terms of $V_0$, $R$, and $r$. Use it to determine the field, the polarization, the bound charge, and the free charge distribution on the sphere.

We know the potential inside the sphere is constant and has the same value as on the surface $V_0$. Outside the sphere, the potential is given by the potential of a point charge with the same charge:
\begin{align*}
V = \frac{Q}{4\pi\epsilon_0 r} = \frac{R}{r}V_0
\end{align*}
The electric field outside the sphere is therefore found from the gradient of this potential as
\begin{align*}
\va{E} = -\grad{V} = -\pdv{r}\qty(\frac{R}{r}V_0)\rhat = \frac{R}{r^2}V_0\rhat
\end{align*}
Thus, the polarization in the dielectric is found in terms of the electric field and susceptibility to be
\begin{align*}
\va{P} = \epsilon_0\chi_e\va{E} = \frac{\epsilon_0\chi_eR}{r^2}V_0\rhat
\end{align*}
Hence, the bound surface and volume charge densities are
\begin{align*}
\sigma_b = \va{P} \vdot (-\rhat) &= -\frac{\epsilon_0\chi_eV_0}{R} \\
\rho_b = -\div{\va{P}} &= -\frac{1}{r^2}\pdv{r}\qty(r^2\frac{\epsilon_0\chi_eR}{r^2}) = 0
\end{align*}
Notice that there is no surface charge in the place not bordering the sphere since the normal vector to the plane and $\rhat$ are perpendicular. Finally, we find the free charge by integrating the electric displacement over the lower and upper hemispheres of the sphere
\begin{align*}
Q_{f,z<0} &= \oint \epsilon\va{E} \vdot \dd{\va{a}} = 2\pi R^2\epsilon\frac{V_0}{R} \\
Q_{f,z>0} &= 2\pi R^2\epsilon_0\frac{V_0}{R}
\end{align*}
Thus, the free charge densities are 
\begin{align*}
\sigma_{f,z<0} &= \epsilon_0\qty(1+\chi_e)\frac{V_0}{R} \\
\sigma_{f,z>0} &= \epsilon_0\frac{V_0}{R} 
\end{align*}

b) Show that the resulting charge configuration would indeed produce the potential $V(r)$.

We can find the potential by integrating over the charge density of the configuration, which is just the sum of the free and bound surface charge densities:
\begin{align*}
\sigma = \sigma_f + \sigma_b
\end{align*}
For $z > 0$, this is just $\epsilon_0\frac{V_0}{R}$ and for $z < 0$ this is
\begin{align*}
\epsilon_0\qty(1+\chi_e)\frac{V_0}{R} - \frac{\epsilon_0\chi_eV_0}{R} = \epsilon_0\frac{V_0}{R}
\end{align*}
Notice, though, that this is just the surface charge density produced in the presence of no dielectric, so the total surface charge density is constant over the sphere. Furthermore, by integrating over the surface of the sphere, we will arrive at the potential produced by the conductor in the absence of the dielectric.

c) Appeal to the uniqueness theorem in Prob. 4.38 to complete the argument.

Because we have found a potential satisfying the boundary conditions, it must be the field for the given configuration. Thus, $V(r)$ is the field with the dielectric in place.

d) Could you solve the configurations in Fig. 4.36 with the same potential? If not, explain \textit{why}.

We should be able to solve the configurations in the Fig. 4.36. In each configuration the critical pieces are the same, save for the fraction of the sphere covered by the dielectric, which is divided out when finding the free surface charge density.


\end{document}
