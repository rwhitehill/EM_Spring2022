\documentclass[12pt,a4paper]{article}

\input{../../preamble_files/packages}
\input{../../preamble_files/scriptr}
\input{../../preamble_files/siunits}
\input{../../preamble_files/vectors}
\input{../../preamble_files/figures}
\input{../../preamble_files/references}
\input{../../preamble_files/empheq}

\pagestyle{fancy}
\lhead{Richard Whitehill}
\chead{PHYS 631 -- HW X}
\rhead{05/10/22}
\cfoot{\thepage \hspace{1pt} of \pageref{LastPage}}

\newcommand{\prob}[2]{\textbf{#1)} #2}

\setlength{\parskip}{\baselineskip}
\setlength{\parindent}{0pt}

\begin{document}

\prob{7.28}{
Find the energy stored in a section of length $l$ of a long solenoid (radius $R$, current $I$, $n$ turns per unit length),
}

a) using Eq. 7.30

We have
\begin{eqbox}
    W = \frac{1}{2} L I^2 = \frac{1}{2} \mu_0 n^2 \pi R^2 l I^2 
.\end{eqbox}
where the inductance of the solenoid is 
\begin{align*}
    L = \mu_0 n^2 \pi R^2 l
.\end{align*}

b) using Eq. 7.31

Eq. 7.31 says that 
\begin{align*}
    W = \frac{1}{2} \oint \left( \va{A} \vdot \va{I} \right) \dd{l} 
.\end{align*}
The vector potential of a solenoid is 
\begin{align*}
    \va{A} = \frac{\mu_0 n I}{2} \frac{R^2}{s} \phihat
.\end{align*}
For the integral, we evaluate this at $s = R$, giving
\begin{eqbox}
    W = \frac{1}{2} \left( \frac{\mu_0 n I R}{2} \right) \left( 2 \pi R \right) = \frac{1}{2} \mu_0 n^2 \pi R^2 l I^2
.\end{eqbox}

c) using Eq. 7.35

Another method gives
\begin{eqbox}
    W = \frac{1}{2 \mu_0} \int B^2 \dd[3]{\va{r}} = \frac{1}{2 \mu_0}\left( \mu_0 n I \right)^2 \int \dd[3]{\va{r}} = \frac{1}{2} \mu_0 n^2 \pi R^2 l I^2
.\end{eqbox}
Notice that the magnetic field vanishes outside the solendoid, so we only have to compute the integral over the volume of the piece of solenoid.

d) using Eq. 7.34

Finally, we have
\begin{align*}
    W = \frac{1}{2 \mu_0}\left[ \int_{\mathcal{V}} B^2 \dd[3]{\va{r}} - \oint_{\mathcal{S}} \left( \va{A} \cross \va{B} \right) \vdot \dd{\va{a}} \right]
.\end{align*}
Notice that $\va{A} = A \phihat$ and $\va{B} = B \zhat$, and $(\phihat \cross \zhat) \vdot \shat = 0$, so the second integral gives zero, which is expected since the first integral is exactly that evaluated in part (c).

\prob{7.44}{
    In a \textbf{perfect conductor}, the conductivity is infinite, so $\va{E} = 0$, and any net charge resides on the surface (just as it does for an \textit{im}perfect conductor, in electro\textit{statics}).
}

a) Show that the magnetic field is constant ($\partial \va{B} / \partial t = 0$), inside a perfect conductor.

From Maxwell, we have
\begin{align*}
    \curl{E} = \pdv{\va{B}}{t}
.\end{align*}
Since, $E \equiv 0$ in the conductor, it follows that
\begin{eqbox}
    \pdv{\va{B}}{t} = 0
\end{eqbox}
inside the conductor as well.

b) Show that the magnetic flux through a perfectly conducting loop is constant.

If we have a perfectly conducting loop, then the magnetic field is constant inside, meaning that 
\begin{eqbox}
    \mathcal{E} = -\pdv{\Phi}{t} = - \int \pdv{\va{B}}{t} \vdot \dd{\va{a}} = 0
.\end{eqbox}

A \textbf{superconductor} is a perfect conductor with the additional property that the (constant) $\va{B}$ inside is in fact \textit{zero}.

c) Show that the current in a superconductor is confined to the surface.

Another one of Maxwell's equations tells us that 
\begin{align*}
    \curl{B} = \mu_0 \va{J} + \epsilon_0 \mu_0 \pdv{\va{E}}{t}
.\end{align*}
Since inside the superconductor $\va{B} \equiv 0$ and $\va{E} \equiv 0$, then we have
\begin{eqbox}
    \va{J} = 0
,\end{eqbox}
inside the conductor.
This leaves only the possibility of current being contained to the surface of the superconductor since it is not necessarily true that $\va{B}$ has no curl at the surface or that the electric field at the surface of the superconductor does not vary with time.

d) Superconductivity is lost above a certain critical temperature ($T_{c}$), which varies from one material to the other.
Suppose you had a sphere (radius $a$) above its critical temperature, and you held it in a uniform magnetic field $B_0 \zhat$ while cooling it below $T_{c}$.
Find the induced surface current density $\va{K}$, as a function of the polar angle $\theta$.

When the superconductor comes below its critical temperature, the magnetic field inside must be zero.
This can only come from a current restricted to its surface.
Since we have solved a problem where the current is from a rotating surface charge, we can immediately write down the magnetic field produced from this surface current:
\begin{align*}
    \va{B} = \frac{2}{3} \mu_0 \sigma a \va{\omega}
.\end{align*}
Since we want the internal magnetic field to vanish, we must have
\begin{align*}
    \va{B} - B_0 \zhat = 0 \Rightarrow \sigma \omega a = -\frac{2 B_0}{3 \mu_0}
.\end{align*}
For a surface density rotating, we know
\begin{align*}
    \va{K} = \sigma \va{v} = \sigma \va{\omega} \cross \va{r} = \sigma \va{\omega} \va{r} \sin{\theta} \phihat
.\end{align*}
Substituting the result above, we have
\begin{eqbox}
    \va{K} = -\frac{2 B_0}{3 \mu_0} \sin{\theta} \phihat
.\end{eqbox}



\end{document}
