\documentclass[12pt,a4paper]{article}

\input{../../preamble_files/packages}
\input{../../preamble_files/scriptr}
\input{../../preamble_files/siunits}
\input{../../preamble_files/vectors}
\input{../../preamble_files/figures}
\input{../../preamble_files/references}
\input{../../preamble_files/empheq}

\pagestyle{fancy}
\lhead{Richard Whitehill}
\chead{PHYS 631 -- HW W}
\rhead{05/05/22}
\cfoot{\thepage \hspace{1pt} of \pageref{LastPage}}

\newcommand{\prob}[2]{\textbf{#1)} #2}

\setlength{\parskip}{\baselineskip}
\setlength{\parindent}{0pt}

\begin{document}

\prob{7.4}{
Suppose the conductivity of the material separating the cylinders in Ex. 7.2 is not uniform; specifically, $\sigma(s) = k/s$, for some constant $k$.
Find the resistance between the cylinders.
}

\begin{figure}[H]
   \begin{center}
       \def\svgwidth{0.5\linewidth} 
       \input{./prob7-4.pdf_tex}
   \end{center} 
\end{figure}

We note that the current moving from the inside cylinder to the outside cylinder is just 
\begin{align*}
    I = J(s)\left( 2 \pi s L \right)
.\end{align*}
Thus, by Ohm's Law
\begin{align*}
    \va{J} = \sigma \va{E} \Rightarrow \va{E} = \frac{I}{2 \pi k L}
.\end{align*}
Hence, the potential difference between the cylinders is 
\begin{align*}
   V = - \int_{b}^{a} \frac{I}{2 \pi k L} \dd{s} = \frac{I}{2 \pi k L}\left( b - a \right). 
.\end{align*}
This gives that 
\begin{eqbox}
    R = \frac{V}{I} = \frac{b-a}{2 \pi k L}
.\end{eqbox}

\newpage

\prob{7.8}{
A square loop of wire (side $a$) lies on a table, a distance $s$ from a very long straight wire, which carries a current $I$, as shown in Fig. 7.18.
}

\begin{figure}[H]
   \begin{center}
       \def\svgwidth{\linewidth} 
       \input{./prob7-8.pdf_tex}
   \end{center} 
\end{figure}

a) Find the flux of $\va{B}$ through the loop.

The flux is simply
\begin{align*}
    \Phi = \int \va{B} \vdot \dd{\va{a}} = \int \frac{\mu_0 I}{2 \pi s} \dd{s} \dd{z} = a \int_{s}^{s+a} \frac{\mu_0 I}{2 \pi s'} \dd{s'}
,\end{align*}
where we have exchanged $s \rightarrow s'$ inside the integral as a dummy variable.
The integration leaves us with
\begin{eqbox}
    \Phi = \frac{\mu_0 I a}{2 \pi} \ln\left( \frac{s + a}{s} \right)
.\end{eqbox}

b) If someone now pulls the loop directly away from the wire, at speed $v$, what emf is generated?
In what direction (clockwise or counterclockwise) does the current flow?

If we pull the loop away perpendicular to the straight wire at a speed $v$, then $\dv{s}{t} = v$, giving us that 
\begin{eqbox}
    \mathcal{E} = -\dv{\Phi}{t} = \frac{\mu_0 I a}{2 \pi} v \left[ \frac{1}{s + a} - \frac{1}{s} \right] = \frac{\mu_0 I a^2 v}{2 \pi s (s + a)}
.\end{eqbox}

Note that the induced current in the loop is counter-clockwise.
This is because as the loop moves further from the wire, the flux decreases, meaning that the loop must produce a magnetic field pointing out of the page, and by the right-hand rule, this gives us a counterclockwise current.

c) What if the loop is pulled to the $right$ at speed $v$?

This setup is translationally invariant parallel to the straight wire.
That is, the magnetic flux does not change when the loop is moved parallel to the wire.
We can then quickly conclude that the emf is identically zero in this case.

\end{document}
