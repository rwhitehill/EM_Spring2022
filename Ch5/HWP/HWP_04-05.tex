\documentclass[12pt,a4paper]{article}

\input{../../preamble_files/packages}
\input{../../preamble_files/scriptr}
\input{../../preamble_files/siunits}
\input{../../preamble_files/vectors}
\input{../../preamble_files/figures}
\input{../../preamble_files/references}
\input{../../preamble_files/empheq}

\pagestyle{fancy}
\lhead{Richard Whitehill}
\chead{PHYS 631 -- HW P}
\rhead{4/05/07}
\cfoot{\thepage \hspace{1pt} of \pageref{LastPage}}

\newcommand{\prob}[2]{\textbf{#1)} #2}

\setlength{\parskip}{\baselineskip}
\setlength{\parindent}{0pt}

\begin{document}

\prob{5.1}{A particle of charge $q$ enters a region of uniform magnetic field $\va{B}$ (pointing \textit{into} the page). The field deflects the particle a distance $d$ above the original line of flight, as shown in Fig. 5.8. Is the charge positive or negative? In terms of $a$, $d$, $B$, and $q$, find the momentum of the particle.}

The magnetic force is given as a cross product of the velocity of a charge and the magnetic field which it is immersed in:
\begin{align*}
\va{F} = q\va{v} \cross \va{B}
\end{align*}

We can determine the sign of the charge on $q$ by seeing if the direction of the force is parallel or anti-parallel to the direction of $\va{v} \cross \va{B}$. Using the right hand rule, we see that the force points up, implying that the charge must be positive.

Now, we know that a charge moving in a constant magnetic field with velocity perpendicular to the field undergoes circular motion:
\begin{align*}
\begin{cases}
x = r\cos(\frac{qB}{m}t) \\
y = r\sin(\frac{qB}{m}t) - r
\end{cases}
\end{align*}
We can solve for the radius of motion using the point $(d,a)$, giving
\begin{align*}
r = \frac{a^2 + d^2}{2d}
\end{align*}
Hence, using basic facts about centripetal motion, we have
\begin{align*}
F = \frac{mv^2}{r} = qvB \\
\Rightarrow p = qBr
\end{align*}
giving
\begin{eqbox}
p = \frac{qB(a^2 + d^2)}{2d}
\end{eqbox}

\prob{5.5}{A current $I$ flows down a wire of radius $a$.}

a) If it is uniformly distributed over the surface, what is the surface current density $K$?

The length of the strip perpendicular to the flow of current is $2 \pi a$, and since the current is uniform over the surface, then 
\begin{eqbox}
K = \frac{I}{2 \pi a}
\end{eqbox}

b) If it is distrubuted in such a way that the volume current density is inversely proportional to the distance from the axis, what is $J(s)$?

We can write
\begin{align*}
J = \frac{k}{s}
\end{align*}

find $k$ in terms of the total current as follows:
\begin{align*}
I = \int \va{J} \vdot \dd{\va{a}} = \int \frac{k}{s} s\dd{\phi}\dd{s} = k (2\pi) a \\
\Rightarrow k = \frac{I}{2 \pi a}
\end{align*}
Hence,
\begin{eqbox}
J(s) = \frac{I}{2 \pi a s}
\end{eqbox}


\end{document}
