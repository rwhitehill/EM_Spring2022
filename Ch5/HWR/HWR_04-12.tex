\documentclass[12pt,a4paper]{article}

\input{../../preamble_files/packages}
\input{../../preamble_files/scriptr}
\input{../../preamble_files/siunits}
\input{../../preamble_files/vectors}
\input{../../preamble_files/figures}
\input{../../preamble_files/references}
\input{../../preamble_files/empheq}

\pagestyle{fancy}
\lhead{Richard Whitehill}
\chead{PHYS 631 -- HW R}
\rhead{04/12/22}
\cfoot{\thepage \hspace{1pt} of \pageref{LastPage}}

\newcommand{\prob}[2]{\textbf{#1)} #2}

\setlength{\parskip}{\baselineskip}
\setlength{\parindent}{0pt}

\begin{document}

\prob{5.14}{A steady current $I$ flows down a long cylindrical wire of radius $a$. Find the magnetic field, both inside and outside the wire, if}

a) The current is uniformly distributed over the outside surface of the wire.

Ampere's law tells us that $\oint \va{B} \vdot \dd{\va{l}} = \mu_0 I$.
Obviously then, the magnetic field is zero inside the wire since there is no enclosed current.

Outside, though, if we enclose the wire with a loop of radius $r$, then
\begin{align*}
    B(2 \pi r) = \mu_0 I \implies B = \frac{\mu_0 I}{2 \pi r}
.\end{align*}

b) The current is distributed in such a way that $J$ is proportional to $s$ , the distance from the axis.

We have already written the current density, which is inversely proportional to the distance from the center of the cylinder, as $J = \frac{I}{2 \pi a s}$.
Hence, inside
\begin{align*}
    B(2 \pi r) &= \mu_0 \int_{0}^{r} \frac{I}{2 \pi a s} \left( 2 \pi s \dd{s} \right) \\
    B = \frac{\mu_0 I }{2 \pi a}
,\end{align*}
and outside we have the same field as in part (a) since the total enclosed current is $I$ .

\prob{5.10}{}

a) Find the density $\rho$ of mobile charges in a piece of copper, assuming each atom contributes one free electron.

The computation here essentially amounts to unit conversions.
By definition, the density of mobile charges is just the total mobile charge per unit volume, or
\begin{align*}
    \rho = \frac{Q}{V}
.\end{align*}
We do not know $Q$ or $V$, but we can write $Q = ZNe$ as the number of mobile electrons contributed per atom on average times the number of atoms and the charge of the electron.
For the volume, we can empirically measure the density of copper: $V = \frac{m}{D}$, where $m$ is the total mass of copper and $D$ is the density of copper (reluctantly used since $\rho$ is already taken).
Now, we also do not know the mass or number of copper atoms, but we can write $m = M n$ and $N = n N_{A}$, where $n$ is the number of moles of copper, $M$ is the molar mass of copper, and $N_{A}$ is Avogadro's number.
Putting it all together gives
\begin{align*}
    \rho = \frac{ZeDN_{A}}{M} = 1.36 \times 10^{10} \text{ \si{\coulomb\per\metre\cubed}}
\end{align*}
for copper.

b) Calculate the average electron velocity in a copper wire 1 \si{\milli\metre} in diameter, carrying a current of 1 \si{\ampere}.

We know that
\begin{align*}
    I = \int \rho \va{v} \vdot \dd{\va{a}} = \rho v \left( \pi R^2 \right) \\
    v = \frac{4\pi}{\rho d^2} = 9.24 \times 10^{-4}
.\end{align*}

This is about an order of magnitude slower than snail speed, which is roughly 0.03 mph.

Note that all the necessary constants were found with a simple Google search.


\end{document}
