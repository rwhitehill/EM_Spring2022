\documentclass[12pt,a4paper]{article}

\input{../../preamble_files/packages}
\input{../../preamble_files/scriptr}
\input{../../preamble_files/siunits}
\input{../../preamble_files/vectors}
\input{../../preamble_files/figures}
\input{../../preamble_files/references}
\input{../../preamble_files/empheq}

\pagestyle{fancy}
\lhead{Richard Whitehill}
\chead{PHYS 631 -- HW T}
\rhead{04/21/22}
\cfoot{\thepage \hspace{1pt} of \pageref{LastPage}}

\newcommand{\prob}[2]{\textbf{#1)} #2}

\setlength{\parskip}{\baselineskip}
\setlength{\parindent}{0pt}

\begin{document}

\prob{6.1}{Calculate the torque exerted on the square loop shown in Fig. 6.6 due to the circular loop (assume $r$ is much larger then $a$ or $b$). If the square loop is free to rotate, what will its quilibrium orientation be?}

\prob{6.5}{A uniform current density $\va{J} = J_0 \zhat$ fills a slab straddling the $yz$ plane, from $x = -a$ to $x = +a$.
A magnetic dipole $\va{m} = m_0 \xhat$ is situated at the origin.}

a) Find the force on the dipole, using Eq. 6.3.

b) Do the same for a dipole pointing in the $y$ direction: $\va{m} = m_0 \\hat{y}$.

c) In the \textit{electrostatic} case, the expressions $\va{F} = \grad \left( \va{p} \vdot \va{E} \right)$ and $\va{F} = \left( \va{p} \vdot \grad \right) \va{E}$ are equivalent (prove it), but this is \textit{not} the case for the magnetic analogs (explain why).
As an example, calculate $\left( \va{m} \vdot \grad \right) \va{B}$ for the configurations in (a) and (b).

\end{document}
